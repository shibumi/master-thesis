\documentclass[titlepage]{report}
\usepackage[backend=biber,style=numeric]{biblatex}
\addbibresource{literature.bib}
\usepackage{caption}
\usepackage{subcaption}
\usepackage{graphicx}
\usepackage[utf8]{inputenc}
\usepackage[T1]{fontenc}
\usepackage{url}
\usepackage{hyphenat}
\usepackage{glossaries}
\usepackage{array}
\usepackage{calc}
\usepackage{booktabs}
\usepackage{hyperref}
\usepackage{listings}
% \usepackage{xcolor} https://tex.stackexchange.com/q/466147
\usepackage{bytefield}
\usepackage{float}
\usepackage{eurosym}
\usepackage{tabu}
\usepackage{caption}
\lstset{%
    frame=tb,
    tabsize=4,
    numbers=left,
    breaklines=true,
}
\setcounter{biburllcpenalty}{9001}
\makeglossaries{}
\newglossaryentry{ima}
{%
    name={IMA},
    description={Computer network airborne system},
    first={Integrated Modular Avionics(IMA)},
    long={Integrated Modular Avionics}
}
\newglossaryentry{apex}
{%
    name={APEX},
    description={APplication/EXecutive Interface},
    first={APplication/EXecutive Interface (APEX)},
    long={APplication/Executive Interface}
}
\newglossaryentry{coex}
{%
    name={COEX},
    description={COre/EXecutive Interface},
    first={COre/EXecutive Interface (APEX)},
    long={COre/Executive Interface}
}
\newglossaryentry{arinc}
{%
    name={ARINC},
    description={Aeronautical Radio Incorporated},
    first={Aeronautical Radio Incorporated (ARINC)},
    long={Aeronautical Radio Incorporated}
}
\newglossaryentry{aeec}
{%
    name={AEEC},
    description={Airlines Electronic Engineering Committee},
    first={Airlines Electronic Engineering Committee (AEEC)},
    long={Airlines Electronic Engineering Committee}
}
\newglossaryentry{lru}
{%
    name={LRU},
    description={Line-Replaceable Unit},
    first={Line-Replaceable Unit (LRU)},
    long={Line-Replaceable Unit},
    plural={LRUs},
    firstplural={Line-Replacable Units (LRUs)}
}
\newglossaryentry{lrm}
{%
    name={LRM},
    description={Line-Replaceable Module},
    first={Line-Replaceable Module (LRM)},
    long={Line-Replaceable Module},
    plural={LRMs},
    firstplural={Line-Replacable Modules (LRMs)}
}
\newglossaryentry{arinc629}
{%
    name={ARINC 629},
    description={ARINC standard for a global data bus in aviation},
    first={ARINC 629 standard},
    long={ARINC 629 standard}
}
\newglossaryentry{arinc650}
{%
    name={ARINC 650},
    description={ARINC standard for IMA packaging and interfaces},
    first={ARINC 650 standard},
    long={ARINC 650 standard}
}
\newglossaryentry{arinc651}
{%
    name={ARINC 651},
    description={ARINC report that provides guidelines for a maintenance strategy for IMA-equipped airplanes},
    first={ARINC 651 report},
    long={ARINC 651 report}
}
\newglossaryentry{arinc653}
{%
    name={ARINC 653},
    description={ARINC standard for space and time partitioning},
    first={ARINC 653 standard},
    long={ARINC 653 standard}
}
\newglossaryentry{arinc659}
{%
    name={ARINC 659},
    description={ARINC standard for a backplane data bus},
    first={ARINC 659 standard},
    long={ARINC 659 standard}
}
\newglossaryentry{RTCA}
{%
    name={RTCA},
    description={Radio Technical Commission for Aeronautics (RTCA)},
    first={Radio Technical Commission for Aeronautics (RTCA)},
    long={Radio Technical Commission for Aeronautics}
}
\newglossaryentry{do178b}
{%
    name={DO-178B},
    description={Certification for safety-critical software by the RTCA},
    first={DO-178B},
    long={DO-178B}
}
\newglossaryentry{do178c}
{%
    name={DO-178C},
    description={Certification for safety-critical software by the RTCA (replaces DO-178B)},
    first={DO-178C},
    long={DO-178C}
}
\newglossaryentry{us}
{%
    name={US},
    description={United States of America. Short: United States},
    first={United States (US)},
    long={United States}
}
\newglossaryentry{io}
{%
    name={I/O},
    description={input and output},
    first={input and output (I/O)},
    long={input and output}
}
\title{From the Cloud to the Clouds: Taking Integrated Modular Avionics on a New Level with Cloud-Native Technologies}
\author{Christian Rebischke\\
Clausthal University of Technology\\
Student ID: 432108 \\
E-Mail: Christian.Rebischke@tu-clausthal.de}
\begin{document}
\maketitle
\chapter*{Acknowledgement}
\chapter*{Statutory Declaration}
This master thesis is submitted in partial fulfilment of the requirements for the Clausthal
University of Technology. I hereby declare that this dissertation is my own work and
contains nothing which is the outcome of work done in collaboration with others,
except as specified in the text and acknowledgements. The contributions of any other
supervisors to this thesis are made with specific reference.
\\
\\
Clausthal-Zellerfeld, \today
\\
\\
Christian Rebischke
\chapter*{Abstract}
The goal of this scientific work is to apply transfer knowledge from the cloud computing area to avionics and to
contribute to a more heterogeneous research picture. The focus of this work lies in particular in the transformation
of avionics architectures from a federated system to an integrated system, as well as its future development.
The challenges and solutions of known architectures will be analyzed and compared with new
achievements in cloud computing. In particular
the Service Orientated Architecture (SOA) approach plays a role in this comparison, as well as its
reliable, secure and cost-effective deployment in airplanes, drones or spaceships.
The master thesis is structured as follows: In the introduction, the classification of the thesis is repeated
and put in context with the current state of the art. Then, in the second chapter, the path from a federated
avionics architecture to an integrated system will be shown and its problems, challenges
and ideas will get isolated. This gained information is subsequently being compared with current cloud native technologies
and potential solutions for these subject will be proposed.

\chapter*{Zusammenfassung}
Ziel dieser wissenschaftlichen Arbeit ist es Transferwissen aus dem Cloud Computing Bereich auf die Avionik
anzuwenden und dazu zu einem heterogeneren Forschungsbild beizutragen. Im Fokus der Arbeit liegt insbesondere
der Weg der Avionik Architekturen von einem föderierten System hin zu einem integrierten System, sowie dessen
zukünftige Weiterentwicklung. Dabei sollen die Herausforderungen und Lösungen von bekannten Architekturen
analysiert und mit neuen Errungenschaften aus dem Cloud Computing Bereich verglichen werden. Insbesondere
der Service Orientierted Architecture (SOA) Ansatz spielt in diesem Vergleich eine Rolle, sowie dessen
zuverlässige, sichere und kostengünstige Einsatzmöglichkeiten in Flugzeugen, Drohnen oder Raumschiffen.
Die Masterarbeit ist wie folgt gegliedert: In der Einleitung wird die Einordnung der Arbeit wiederholt
und in einen Zusammenhang mit der Gegenwart gestellt. Im Zweiten Kapitel wird dann der Weg von einer föderierten
Avionik Architektur zu einem integrierten System beleuchtet und dessen Probleme, Herausforderungen
und Ideen isoliert. Diese gewonnenen Informationen werden nachfolgend aktuellen Cloud Native Technologien
gegenüber gestellt und potentielle Lösungen vorgeschlagen.

\tableofcontents
\chapter{Introduction}
The number of performed flights by global airline industries increased from 23.8 million flights (2004)
to 38.9 million flights (2019)\cite{STATISTA}. This growing number of performed flights puts an enormous
pressure on the global aviation industry as a whole. The permanent price pressure lead to demands of
cheaper, lighter and smaller flight components\cite{prisaznuk1992integrated}. Every inch and every gramm 
counts in the global business of civil aviation, because every inch less means one possible paying customer 
more on the plane and every gramm less means less expensive fuel demands for the flight. As a result
the aviation industry has been actively searching for solutions for the aforementioned problems.
The ongoing digitalization brought different ideas for technological improvements in the aviation sector.
This master thesis aims to highlight the main problems and a few of these ideas. Furthermore
it tries to find related problems in the field of cloud computing and tries to connect these
problems with their possible solutions with the known problems in the aviation industry. The ultimate
goal of this thesis is to provide alternative research to the existing proposed solutions for the explored
problems in the flight industry and to create a more heterogenous field of view on the current research
about architectures in avionics.
\chapter{Integrated Modular Avionics}

One of these technologies is \gls{ima}. \gls{ima} promises a price, space and weight cut, leading to more 
space and weight for customers, goods or other equipment (for example weapon payload in the military sector)
via using fewer, more centralized processing units. The aim of this master thesis is the exploration
of the concepts and challenges behind \gls{ima} and comparing it to existing solutions from the new forming
Cloud-Native initiative driven by cloud companies.  
\chapter{Related Work}
\nocite{*}
\printbibliography{}
\lstlistoflistings{}
\listoftables{}
\listoffigures
\glsaddallunused
\printglossary{}
\end{document}
