\documentclass[conference]{IEEEtran}
\usepackage[backend=biber,style=numeric]{biblatex}
\usepackage[utf8]{inputenc}
\usepackage[T1]{fontenc}
\usepackage{graphicx}
\usepackage{caption}
\usepackage{subcaption}
\addbibresource{literature.bib}

\begin{document}
\title{From the Cloud to the Edge: Applying Cloud Native Technologies to the Aviation Sector}
\author{\IEEEauthorblockN{Christian Rebischke}
\IEEEauthorblockA{Clausthal University of Technology\\
Christian.Rebischke@tu-clausthal.de}}

\maketitle

\begin{abstract}
Software has become the daily driver for nearly every industry. It can be found
in automotives, ships, production sites, nine to five office jobs and the aviation
sector. The use of software and control engineering lead to self-landing rockets,
artificial intelligence controlled unmanned aerial vehicles and adaptive touch displays
in space capsules like SpaceX's Dragon capsule. Therefore software becomes an
important part in the supply chain of aviation and plays an important role
in flight safety, resiliency and scalability. This proposal summarizes a possible
research topic for a master thesis that tries to connect the dots between Cloud Native
technologies and the aviation sector for revealing areas of application in flight safety, resiliency
and scalability of service oriented architectures in the aviation sector.
\end{abstract}

\IEEEpeerreviewmaketitle

\section*{Introduction}
In the year of 2015 the Linux Foundation founded along with several partners the Cloud Native Computing
Foundation (CNCF) to help advance container technologies and align the tech industry
around it's evolution\cite{CNCFFounding}. Since then the project has accomplished
a founding of over 15 billion US dollars and an accumulated market capitalization 
over all contributing companies of over 15 trillion US dollars\cite{CNCFLandscape}.
In 2019 the project tracked more than 1200 projects, products and companies from different
industry sectors\cite{CNCFAnnualReport2019}. The scope of these projects reaches from
high available and high scalable container orchestration, over service discovery,
application definition and image builds to continuous integration and continuous delivery.
Single CNCF projects found already its way into the aviation sector. In 2020 the Department
of Defense of the United States of America announced the first deployment of the
container orchestrator and scheduler Kubernetes on the U-2 spy plane. This deployment
of Cloud-Native software gives just one short glimpse on the possibilities for
this young set of technologies.

\section*{Motivation}
Service Oriented Architecture (SOA) made the deployment of software on any possible 
system fairly easy and fast, but it introduced other possible problems. This short
section tries to give a good overview over possible challenges for service oriented
architecture and tries to give examples in the aviation sector.

\subsection{Service Resiliency}
A software service is not a fire-and-forget like military missile. The service
needs constant monitoring to verify its health status based on certain metrics
and if such metrics violate certain conditions a service scheduler should
take measures to change the current status to a healthy status again. Just imagine
a service that provides a REST API for communicating the current plane position
to other services like the traffic collision avoidance system (TCAS). A wrong communicated
result by the service or a failure of the service could have a catastrophic scale.
A monitoring system is needed that checks such service, verifies the state and takes
measures like alerting the pilots or re-scheduling the service. How can such
a system behave, look like and work? How should such a monitoring system look like?

\subsection{Service Security}
Due to the intense usage of the Service Oriented Architecture in the aviation sector
the usage of open source software has grown tremendously and with it grew
the possible inclusion of security vulnerabilities. A secure software supply chain
has become an important part in the production of aviation software. How can we make
sure that we are running the same software in our test environment on the ground
as on our planes in the air? How do we make sure that our software is free of possible
security vulnerabilities and how do we detect those? And how we can defend ourselves
from possible security violations and hackers tamping with our software supply chain?

\subsection{Service Scalability}
The number of planes in the air grew in the last decade (the current Covid-19 pandemic excluded).
This makes it much more difficult to monitor the flight traffic and it introduces additional
cost to flight traffic monitoring agencies. Imagine a company or agency that monitors over
90\% of planes only during the week days. During the weekends such a giantic setup would not be needed.
This introduces high costs to the flight traffic monitoring service. Costs that could have been 
avoided if the flight traffic monitoring service would make use of autoscalable cloud-native technologies.
Modern cloud-native technologies could be able to automatically scale up and down on the current
demand and could adapt on current flight traffic, while maintaining a high service resiliency
and flight safety due to possible backup services.

\section*{Summary and Approach}
The last three subsections gave a good overview over possible fields of application
for Cloud Native technologies. This proposal has introduced questions for the aviation
sector that clearly needs more research. A possible idea for evaluating the mentioned
problems could be adapting a Service Oriented Flight Architecture to the Cloud-Native
world via using Cloud-Native technologies and trying to introduce state-of-the-art 
solutions for problems on the cloud while adapting them to the edge. Cloud technologies
has proven to be highly scalable and resilient, why not trying to gain the same benefits
for the aviation sector? A possible master thesis could determine similarities between
cloud and aviation demands and could show off possible solutions for vice-versa.
The goal could be to build a full secure software supply chain for a Service Oriented
Architecture while not losing the scope of resiliency and scalability.
\nocite{*}
\printbibliography[]{}
\end{document}
